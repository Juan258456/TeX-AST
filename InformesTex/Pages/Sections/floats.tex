\section{Floats} \label{sec:floats}

Floats are non-text items that generally can not be split over a page.  They also have captions and can be numbered for reference.  Primarily these are figures and tables but authors can define their own. \latex\ tries to place a float where indicated in the manuscript but will move it later if there is not enough room at that location, hence the term ``float''.

Authors are encouraged to embed their tables and figures within the text as they are mentioned.  Editors and the vast majority of referees find it much easier to read a manuscript with embedded figures and tables.

Depending on the number of floats and the particular amount of text and equations present in a manuscript the ultimate location of any specific float can be hard to predict prior to compilation. It is recommended that authors \textbf{not} spend significant time trying to get float placement perfect for peer review.  The AAS Journal's publisher has sophisticated typesetting software that will produce the optimal layout during production.

Note that authors of Research Notes are only allowed one float, either one table or one figure.

For authors that do want to take the time to optimize the locations of their floats there are some techniques that can be used.  The simplest solution is to placing a float earlier in the text to get the position right but this option will break down if the manuscript is altered.  A better method is to force \latex\ to place a float in a general area with the use of the optional {\tt\string [placement specifier]} parameter for figures and tables. This parameter goes after {\tt\string \begin\{figure\}}, {\tt\string \begin\{table\}}, and {\tt\string \begin\{deluxetable\}}.  The main arguments the specifier takes are ``h'', ``t'', ``b'', and ``!''.  These tell \latex\ to place the float \underline{h}ere (or as close as possible to this location as possible), at the \underline{t}op of the page, and at the \underline{b}ottom of the page.  The last argument, ``!'', tells \latex\ to override its internal method of calculating the float position.  A sequence of rules can be created by using multiple arguments.  For example, {\tt\string \begin\{figure\}[htb!]} tells \latex\ to try the current location first, then the top of the page and finally the bottom of the page without regard to what it thinks the proper position should be.  Many of the tables and figures in this article use a placement specifier to set their positions.

Note that the \latex\ {\tt\string tabular} environment is not a float.  Only when a {\tt\string tabular} is surrounded by {\tt\string\begin\{table\}} ...  {\tt\string\end\{table\}} is it a true float and the rules and suggestions above apply.

In AASTeX all deluxetables are float tables and thus if they are longer than a page will spill off the bottom. Long deluxetables should begin with the {\tt\string\startlongtable} command. This initiates a longtable environment.  Authors might have to use {\tt\string\clearpage} to isolate a long table or optimally place it within the surrounding text.

\subsection{Tables} \label{subsec:tables}

Tables can be constructed with \latex's standard table environment or the \aastex's deluxetable environment. The deluxetable construct handles long tables better but has a larger overhead due to the greater amount of defined mark up used set up and manipulate the table structure.  The choice of which to use is up to the author.  Examples of both environments are used in this manuscript. 

Tables longer than 200 data lines and complex tables should only have a short example table with the full data set available in the machine readable format.  The machine readable table will be available in the HTML version of the article with just a short example in the PDF. Authors are required to indicate in the table comments that the data in machine readable format in the full article.  Authors are encouraged to create their own machine readable tables using the online tool at \url{http://authortools.aas.org/MRT/upload.html}.

\aastex\ v6 introduced five new table features that were designed to make
table construction easier and the resulting display better for AAS Journal
authors. The items are:

\begin{enumerate}
\item Declaring math mode in specific columns,
\item Column decimal alignment, 
\item Automatic column header numbering,
\item Hiding columns, and
\item Splitting wide tables into two or three parts.
\end{enumerate}

Full details on how to create each of these special table types are given in the guidelines at \url{http://journals.aas.org/authors/aastex.html}.

\subsubsection{Extremely wide tables}

Since the AAS Journals are now all electronic with no print version there is no reason why tables can not be as wide as authors need them to be. For wide tables, the full table will almost always be available in machine readalbe format with just an example in the article but how is an example created for a wide table?

There are two ways to create examples for wide tabular data sets. The first is to to break a table into two or three components so that it flows down a page by invoking a new table type, splittabular or splitdeluxetable. Within these tables a new ``B'' column separator is introduced.  Much like the vertical bar option, ``$\vert$'', that produces a vertical table lines the new ``B'' separator indicates where to \underline{B}reak a table.  Up to two ``B''s may be included.

Table 1 shows how to split a wide deluxetable into three parts with
the {\tt\string\splitdeluxetable} command.  The {\tt\string\colnumbers}
option is on to show how the automatic column numbering carries through the
second table component.

\begin{splitdeluxetable*}{lccccBcccccBcccc}
\tabletypesize{\scriptsize}
\tablewidth{0pt} 
\tablecaption{Measurements of Emission Lines: two breaks \label{tab:deluxesplit}}
\tablehead{
\colhead{Model} & \colhead{Component}& \colhead{Shift} & \colhead{FWHM} &
\multicolumn{10}{c}{Flux} \\
\colhead{} & \colhead{} & \colhead{($\rm
km~s^{-1}$)}& \colhead{($\rm km~s^{-1}$)} & \multicolumn{10}{c}{($\rm
10^{-17}~erg~s^{-1}~cm^{-2}$)} \\
\cline{5-14}
\colhead{} & \colhead{} &
\colhead{} & \colhead{} & \colhead{Ly$\alpha$} & \colhead{N\,{\footnotesize
V}} & \colhead{Si\,{\footnotesize IV}} & \colhead{C\,{\footnotesize IV}} &
\colhead{Mg\,{\footnotesize II}} & \colhead{H$\gamma$} & \colhead{H$\beta$}
& \colhead{H$\alpha$} & \colhead{He\,{\footnotesize I}} &
\colhead{Pa$\gamma$}
} 
\colnumbers
\startdata 
{       }& BELs& -97.13 &    9117$\pm      38$&    1033$\pm      33$&$< 35$&$<     166$&     637$\pm      31$&    1951$\pm      26$&     991$\pm 30$&    3502$\pm      42$&   20285$\pm      80$&    2025$\pm     116$& 1289$\pm     107$\\ 
{Model 1}& IELs& -4049.123 & 1974$\pm      22$&    2495$\pm      30$&$<     42$&$<     109$&     995$\pm 186$&      83$\pm      30$&      75$\pm      23$&     130$\pm      25$& 357$\pm      94$&     194$\pm      64$& 36$\pm      23$\\
{       }& NELs& \nodata &     641$\pm       4$&     449$\pm 23$&$<      6$&$<       9$&       --            &     275$\pm      18$& 150$\pm      11$&     313$\pm      12$&     958$\pm      43$&     318$\pm 34$& 151$\pm       17$\\
\hline
{       }& BELs& -85 &    8991$\pm      41$& 988$\pm      29$&$<     24$&$<     173$&     623$\pm      28$&    1945$\pm 29$&     989$\pm      27$&    3498$\pm      37$&   20288$\pm      73$& 2047$\pm     143$& 1376$\pm     167$\\
{Model 2}& IELs& -51000 &    2025$\pm      26$& 2494$\pm      32$&$<     37$&$<     124$&    1005$\pm     190$&      72$\pm 28$&      72$\pm      21$&     113$\pm      18$&     271$\pm      85$& 205$\pm      72$& 34$\pm      21$\\
{       }& NELs& 52 &     637$\pm      10$&     477$\pm 17$&$<      4$&$<       8$&       --            &     278$\pm      17$& 153$\pm      10$&     317$\pm      15$&     969$\pm      40$&     325$\pm 37$&
     147$\pm       22$\\
\enddata
\tablecomments{This is an example of how to split a deluxetable. You can
split any table with this command into two or three parts.  The location of
the split is given by the author based on the placement of the ``B''
indicators in the column identifier preamble.  For more information please
look at the new \aastex\ instructions.}
\end{splitdeluxetable*}

The second way is to create a "descriptive" table instead. This type of table only provides information about the columns rather than the data itself. Table 2 shows an example of this type of table using the same columns as in Table 1. Since these types of tables always have a machine readable component, this table uses the {\tt\string \digitalasset}\ command to highlight this fact.

\begin{deluxetable*}{rlll}
\digitalasset
\tablewidth{0pt}
\tablecaption{Descriptive version of the "Measurements of Emission Lines" table \label{tab:description}}
\tablehead{
\colhead{Number} & \colhead{Units} & \colhead{Label} & \colhead{Explanation}
}
\startdata
1 & --- & Model & Model identifier \\
2 & --- & Component & Component identifier \\
3 & $\rm km~s^{-1}$ & Shift & Line shift \\
4 & $\rm km~s^{-1}$ & FWHM & Line Full-Width at Half-Maximum \\
5 & $\rm 10^{-17}~erg~s^{-1}~cm^{-2}$ & Ly$\alpha$ & Ly$\alpha$ line flux \\
6 & $\rm 10^{-17}~erg~s^{-1}~cm^{-2}$ & N\,{\footnotesize V} & N\,{\footnotesize V} line flux \\
7 & $\rm 10^{-17}~erg~s^{-1}~cm^{-2}$ & Si\,{\footnotesize IV} & Si\,{\footnotesize IV} line flux \\
8 & $\rm 10^{-17}~erg~s^{-1}~cm^{-2}$ & C\,{\footnotesize IV} & C\,{\footnotesize IV} line flux \\
9 & $\rm 10^{-17}~erg~s^{-1}~cm^{-2}$ & Mg\,{\footnotesize II} & Mg\,{\footnotesize II} line flux \\
10 & $\rm 10^{-17}~erg~s^{-1}~cm^{-2}$ & H$\gamma$ & H$\gamma$ line flux \\
11 & $\rm 10^{-17}~erg~s^{-1}~cm^{-2}$ & H$\beta$ & H$\beta$ line flux \\
12 & $\rm 10^{-17}~erg~s^{-1}~cm^{-2}$ & H$\alpha$ & H$\alpha$ line flux \\
13 & $\rm 10^{-17}~erg~s^{-1}~cm^{-2}$ & He\,{\footnotesize I} & He\,{\footnotesize I} line flux \\
14 & $\rm 10^{-17}~erg~s^{-1}~cm^{-2}$ & Pa$\gamma$ & Pa$\gamma$ line flux \\
\enddata
\tablecomments{Table 2 is published in its entirety in the electronic 
edition of the {\it Astrophysical Journal}.  A portion is shown here 
for guidance regarding its form and content. The {\tt\string \digitalasset}\ command highlights the Table title to visually indicate to the reader that there is data associated with this table.}
\end{deluxetable*}

\subsection{Figures\label{subsec:figures}}

Authors can include a wide number of different graphics with their articles.
These range from general figures all authors are familiar with
to new enhanced graphics that can only be fully experienced in HTML.  The
later include figure sets, animations and interactive figures.  All
enhanced graphics require a static two dimensional representation in the
manuscript to serve as the example for the reader. All figures should
include detailed and descriptive captions.  These captions are absolutely
critical for readers for whom the enhanced figure is inaccessible either
due to a disability or offline access.  This portion of the article
provides examples for setting up all these types in with the latest version
of \aastex.

\subsection{General figures\label{subsec:general}}

\aastex\ has a {\tt\string\plotone} command to display a figure consisting
of one figure file.  Figure \ref{fig:general} is an example which shows
how AAS Publishing spends author publication charges. For a general figure
consisting of two figure files the {\tt\string\plottwo} command can be
used to position the two image files side by side.

Both {\tt\string\plotone} and {\tt\string\plottwo} take a
{\tt\string\caption} and an optional {\tt\string\figurenum} command to
specify the figure number\footnote{It is better to not use
{\tt\string\figurenum} and let \latex\ auto-increment all the figures. If you
do use this command you need to mark all of them accordingly.}.  Each is
based on the {\tt\string graphicx} package command,
{\tt\string\includegraphics}.  Authors are welcome to use
{\tt\string\includegraphics} along with its optional arguments that control
the height, width, scale, and position angle of a file within the figure.
More information on the full usage of {\tt\string\includegraphics} can be
found at \break
\url{https://en.wikibooks.org/wiki/LaTeX/Importing\_Graphics\#Including\_graphics}.

\subsection{Enhanced graphics}

Enhanced graphics have an example figure to serve as an example for the
reader and the full graphical item available in the published HTML article.
This includes Figure sets, animations, and interactive figures. The 
Astronomy Image Explorer (\url{http://www.astroexplorer.org/}) provides 
access to all the figures published in the AAS Journals since they offered
an electronic version which was in the mid 1990s. You can filter image
searches by specific terms, year, journal, or type. The type filter is 
particularly useful for finding all published enhanced graphics. As of
August 2024 there are over 5600 videos, 2200 figure sets, and 200 interactive
figures. The next sections describe how to include these types of graphics
in your own manuscripts.
