\section{Software and third party data repository citations} \label{sec:cite}

The AAS Journals would like to encourage authors to change software and
third party data repository references from the current standard of a
footnote to a first class citation in the bibliography.  As a bibliographic
citation these important references will be more easily captured and credit
will be given to the appropriate people.

The first step to making this happen is to have the data or software in
a long term repository that has made these items available via a persistent
identifier like a Digital Object Identifier (DOI).  A list of repositories
that satisfy this criteria plus each one's pros and cons are given at \break
\url{https://github.com/AASJournals/Tutorials/tree/master/Repositories}.

In the bibliography the format for data or code follows this format: \\

\noindent author year, title, version, publisher, prefix:identifier\\

\citet{2015ApJ...805...23C} provides a example of how the citation in the
article references the external code at
\doi{10.5281/zenodo.15991}.  Unfortunately, bibtex does
not have specific bibtex entries for these types of references so the
``@misc'' type should be used.  The Repository tutorial explains how to
code the ``@misc'' type correctly.  The most recent .bst file, aasjournalv7.bst, will output bibtex ``@misc'' type properly.

Authors can also use the website \url{https://www.doi2bib.org/} to create a BIBTeX entry for any DOI. Please check the output from this site carefully as its output is only as good as the DOI metadata. Some DOI creators do not provide enough metadata to construct an adequate citation.
