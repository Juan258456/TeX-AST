\section{Author publication charges} \label{sec:pubcharge}

In April 2011 the traditional way of calculating author charges based on 
the number of printed pages was changed.  The reason for the change
was due to a recognition of the growing number of article items that could not 
be represented in print. Now author charges are determined by a number of
digital ``quanta''.  A single quantum is defined as 350 words, one figure, one table,
and one digital asset.  For the latter this includes machine readable
tables, data behind a figure, figure sets, animations, and interactive figures.  The current cost
for the different quanta types is available at 
\url{https://journals.aas.org/article-charges-and-copyright/#author_publication_charges}. 
Authors may use the ApJL length calculator to get a {\tt rough} estimate of 
the number of word and float quanta in their manuscript. The calculator 
is located at \url{https://authortools.aas.org/ApJL/betacountwords.html}.