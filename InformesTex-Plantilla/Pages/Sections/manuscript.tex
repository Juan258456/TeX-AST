\section{Manuscript styles} \label{sec:style}

The default style in \aastex\ v7 is a tight single column style, e.g. 10
point font, single spaced.  The single column style is very useful for
article with wide equations. It is also the easiest to style to work with
since figures and tables, see Section \ref{sec:floats}, will span the
entire page, reducing the need for address float sizing.

To invoke a two column style similar to the what is produced in
the published PDF copy use \\

\noindent {\tt\string\documentclass[twocolumn]\{aastex7\}}. \\

\noindent Note that in the two column style figures and tables will only
span one column unless specifically ordered across both with the ``*'' flag,
e.g. \\

\noindent{\tt\string\begin\{figure*\}} ... {\tt\string\end\{figure*\}}, \\
\noindent{\tt\string\begin\{table*\}} ... {\tt\string\end\{table*\}}, and \\
\noindent{\tt\string\begin\{deluxetable*\}} ... {\tt\string\end\{deluxetable*\}}. \\

\noindent This option is ignored in the {\tt\string onecolumn} style.

All authors should have the {\tt\string linenumbers} style include so that
the compiled PDF has each row numbered in the left margin. Line numbering
is mandatory as it helps reviewers quickly identify locations in the text.

The {\tt\string anonymous} option will prevent the author and affiliations
from being shown in the compiled pdf copy. This option allows the author 
to keep this critical information in the latex file but prevent the reviewer
from seeing it during peer review if dual anonymous review (DAR) is requested. 
Likewise, acknowledgments and author contributions can also be hidden if placed in the {\tt\string\begin\{acknowledgments\}} ... {\tt\string\end\{acknowledgments\}} and {\tt\string\begin\{contribution\}} ... {\tt\string\end\{contribution\}} environments. The use of this option is highly recommended for PSJ submissions. Advice for anonymizing your manuscript for DAR is provided at 
\url{https://journals.aas.org/manuscript-preparation/#dar}.

Another reason to use the {\tt\string\begin\{acknowledgments\}} ... {\tt\string\end\{acknowledgments\}} and {\tt\string\begin\{contribution\}} ... {\tt\string\end\{contribution\}} environments is that the word counter in our peer review system will \textbf{not} count the contents of these environments. If authors put aknowledgments and contribution text in other locations, these words will be counter and authors may be overcharged on their author publication charges.

Multiple style options are allowed, e.g. \\

\noindent {\tt\string\documentclass[linenumbers,trackchanges,anonymous]\{aastex7\}}. \\
